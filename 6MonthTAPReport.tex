\documentclass[11pt, oneside]{article}   	% use "amsart" instead of "article" for AMSLaTeX format
\usepackage{geometry}                		% See geometry.pdf to learn the layout options. There are lots.
\geometry{letterpaper}                   		% ... or a4paper or a5paper or ... 
%\geometry{landscape}                		% Activate for rotated page geometry
\usepackage[parfill]{parskip}    		% Activate to begin paragraphs with an empty line rather than an indent
\usepackage{graphicx}				% Use pdf, png, jpg, or eps§ with pdflatex; use eps in DVI mode
								% TeX will automatically convert eps --> pdf in pdflatex		
\usepackage{amssymb}
\usepackage{color}
\usepackage{subcaption}
\usepackage{caption}
\usepackage[margin=10pt,font=small,labelfont=bf,
labelsep=endash]{caption}


\newcommand{\todo}[1]{ \textcolor{red}{\bf{To Do:} #1}}
\newcommand{\toref}[1]{ \textcolor{blue}{\bf{REFERENCE #1}}}
\newcommand{\future}[1]{ \textcolor{green}{\bf{Future Direction: #1}}}

%SetFonts

%SetFonts


\title{6 Month TAP Report}
\author{By Me :)}
\date{}							% Activate to display a given date or no date

\begin{document}
\maketitle

\section{Motivation}
\subsection{Wounds}

Wounds and their management are an issue both in the UK and globally \cite{Posnett2008the}.
Methods of accelerating wound healing to decrease infection risk are sought, especially in developing countries where wound infection rates are extremely high \cite{Kihla2014risk}. 
Common wound-infecting pathogens such as \textit{Staphylococcus aureus} and \textit{pseudomonas aeruginosa} \cite{Church2006burn, Bowler2001wound} are becoming increasingly resistant to antibiotics, therefore, new wound treatments appropriate for the healthcare budgets of both developed and developing countries are required \cite{Chambers2009waves, Godebo2013multidrug, Howell2005a}.
One technique under investigation is low temperature plasma (LTP), which has shown promise for both bacterial killing and wound healing promotion \cite{Kong2009plasma, Kramer2013suitability, Isbary2012successful, Isbary2010a}.

\subsection{Low Temperature Plasmas in Biomedicine}
Low temperature plasmas (LTP) are weakly ionised plasmas, existing far from thermodynamic equilibrium.
This results in a plasma that has an overall temperature of $\approx$ 25-40$^\circ$C, which can be applied to biological samples without causing thermal damage.

The field of 'plasma medicine' is a rapidly expanding one, with LTP use and research increasing in various applications such as wound healing \cite{Haertel2014nonthermal, Isbary2013nonthermal}, cancer treatment \cite{Hirst2016low, Fridman2007floating}, coagulation \cite{Fridman2006blood, Chen2009blood} and sterilisation \cite{Fridman2006blood, Laroussi2002nonthermal}.
%The interest in LTP for biomedical applications arises from the ability of plasma-produced species being able to influence biological functioning in a range of substrates.
%For example, LTP has been shown to be effective in inactivating bacteria in both planktonic and biofilm forms \cite{Joshi2010control, Pei2012inactivation, Ziuzina2015cold}, whereas it can also have positive influences on eukaryotic cells \cite{Haertel2014nonthermal}.
%The different effects being seen thought to be dose-dependent, with high doses thought to promote killing and lower doses thought to enhance normal physiological function, useful in applications such as wound healing.
Applications such as these result from the abiltiy of LTP to interact with living biological substrates such as bacteria, both in planktonic and biofilm forms \cite{Joshi2010control, Pei2012inactivation, Ziuzina2015cold} and eukaryotic, mammalian cells \cite{Haertel2014nonthermal}.
For sterilisation applications, the properties of low temperature plasma, in particular the low temperature and ability to operate at atmospheric pressure, allow their use both on equipment surfaces \cite{Laroussi2002nonthermal} and also directly on biological tissue, for example chronically infected wounds \cite{Haertel2014nonthermal, Isbary2013nonthermal, Fridman2006blood}.
%
%\todo{Contamination - bacteria at wound site just coz. Colonisation - here, bacteria proliferate but do not cause a host response. Some colonisation may stop healing process, or it may enhance healing as local inflammation will increase perfusion of the area and therefore may help the healing process. Critical colonisation occurs after this (controversially!) and is when the bacterial load increases which may cause chronic inflammation and prevent healing. Transition to infection occurs when the bacterial load overcomes host response and host damage occurs. This transition point depends on the bioburden and the health state of the host.}

\section{Project Aims}
The overall aims of this project are to investigate the use of air plasmas for wound healing applications.
The main themes are:
\begin{enumerate}
\item Characterisation of air plasma - This will involve the development of an air plasma to be characterised in terms of plasma species production, both experimentally and computationally
\item Biological interaction - This will involve trying to understand how the plasma influences different cellular functions in both bacterial and mammalian host cells, by developing \textit{in vitro} biological assays. 
\item Comparison of plasma and biological studies - This should help with the determination of the roles of the different plasma components
\end{enumerate}

\section{Literature Review}
In treating wounds, particularly chronic wounds, there are two main aims.
Firstly, reduction in bacterial load. 
Following injury, wounds become contaminated because bacteria is all around us. 
Proliferation of bacteria can then lead to colonisation of the wound, which may or may not be detrimental to the healing process. 
However, if the wound becomes 'critically colonised', wound healing is thought to be impaired, and infection can follow \cite{Siddiqui2010chronic}. 
Therefore, prevention of infection through reduction in bacterial load would be beneficial.
Secondly, particularly in chronic wounds where the healing process has stalled, promotion of the healing process is desirable.

%When considering wound treatment applications, there are two main motivations for developing plasma treatments.
%Firstly, wounds are often colonised with bacteria which may be preventing the healing process (depending on the degree and type of colonisation), therefore, wound sterilisation by LTP might be beneficial.
%Secondly, wounds can often become chronic, meaning that the healing process has stagnated. In this case, LTP may be able to help reactivation of host cells to promote the healing process.
LTP is able to interact with biological function, and this is likely to be due to its multimodal nature.
LTPs typically contain:
\begin{itemize}
\item Reactive Species
\item Photon Emission
\item Ions
\item Electric Fields
\end{itemize}
The ability of LTPs to influence biological functioning is likely to be due to a combination of the above.
In the following sections, each of the above will be discussed with respect to their roles in LTP-mediated bacterial killing and wound healing promotion.
It is important to note that for each of the plasma components, the proposed mechanism of action discussed below is mainly speculative, following investigation of them directly, not as part of a plasma.


%
%While exact mechanisms of action are not fully understood, there have been investigations into how LTP can influence biological functioning.
%Electrons liberated in the plasma mediate the different plasma components, for example, dissociation of molecules, yielding reactive radical species, emission as a result of electron impact electronic excitation and ionisation which yields ions and allows plasma sustainment.
%It is, therefore, likely that it is a combination of all these components that result in the biological effects seen.
%It is also possible that this multimodal nature may slow or prevent the ability of bacteria to build up resistance to the killing mechanisms.
%This would be an obvious advantage in the era of increasing resistance of bacteria to a variety of antibiotics.
%
% the roles of the different plasma components.
%Briefly, low temperature plasmas are very weakly ionised gases (degree of ionisation $<$ 1\%), with a global temperature of of approximately room temperature ($< 40^{\circ} C$), and dry reactive environment, allowing direct contact with biological substrates.
%Electrons liberated in the plasma mediate the different plasma components, for example, dissociation of molecules, yielding reactive radical species, emission as a result of electron impact electronic excitation and ionisation which yields ions and allows plasma sustainment.
%It is, therefore, likely that the efficacy seen so far using plasmas is due to this complex plasma composition, and also what suggests that bacteria are going to be less likely to build up resistance, due to the multimodal nature of the plasma treatment.
%
%In the following sections, the main plasma components, reactive species, emission and ions and electric fields, will be discussed in the context of their roles in bacterial killing and healing promotion following LTP treatment.

\subsection{Bacterial Killing}
\subsubsection{RONS}
LTP is a highly efficient source of reactive species, produced by mechanisms such as electron impact excitation and dissociation.
The gas present in the plasma determines the types of species that will be produced. 
In the case of air plasma, reactive oxygen and nitrogen species (RONS) are readily produced and these have multiple possible effects on biological substrates.
Important species produced include atomic oxygen (O), ozone (O$_3$), singlet delta oxygen (O$_2$($^1\Delta$)), superoxide (O$_2^-\cdot$), atomic nitrogen (N), hydroxyl radicals (OH), hydrogen peroxide (H$_2$O$_2$) and various nitrogen oxides (NO, NO$_2$ and N$_2$O) \cite{Graves2014low}.

Originally thought to be purely damaging molecules \cite{Harman1955aging}, it is now known that RONS are highly important in normal physiological functioning, acting as signalling molecules and bactericidal agents in immune cells \cite{Thannickal2000reactive}.
However, RONS must be kept at appropriate levels to be non harmful to the host.
At low concentrations they are vital to human health \cite{Fang2004antimicrobial}, however if the concentration of RONS rises excessively, they can become toxic, causing damage to lipids, proteins and DNA \cite{PhamHuy2008free} and lead to cell death.

%Whilst reactive species were thought to be purely damaging molecules produced in the body which only serve to promote disease and ageing \cite{Harman1955aging}, it is now known that their role is far more complex than this, and that reactive species have crucial roles to play in the immune system and in normal physiological signalling \cite{Thannickal2000reactive}.
%At low concentrations, RONS are vital to human health. 
%For example, they are important cell signalling regulating molecules in cells such as fibroblasts, endothelial cells and smooth muscle cells.
%Further to this, they are essential for host defence as they are produced by immune cells during an immune response in order to kill invading pathogens.
%The importance of this function is shown by patients with chronic granulomatous disease (CGD), who cannot synthesise the superoxide radical (O$_2^-\cdot$) in phagocytic immune cells, resulting in multiple, persistent infections \cite{PhamHuy2008free, Fang2004antimicrobial}.
%
%If the concentrations of RONS become excessive and there is an imbalance between radical formation and neutralisation, then they can become dangerous, causing damage to lipids, proteins and DNA \cite{PhamHuy2008free} and lead to cell death.
Considering plasma-produced ROS, these exogenous species can enter cells across the membrane (either by diffusion or transport depending on the species \cite{Bienert2006membrane}), or induce the formation of new ROS intracellularly and exert their action \cite{Haertel2014nonthermal}.
RONS can be visualised inside cells using fluorescent dyes, an example of which is shown by Joshi \textit{et al} \cite{Joshi2011nonthermal}.
In this study, they showed that the level of reactive species present in bacteria (E. Coli), was significantly increased after plasma treatment, compared to untreated controls.
This suggests plasmas can induce oxidative stress in the treated bacteria, which can lead to bacterial death.

RONS attacking lipids, proteins and DNA are believed to be important effector mechanisms for bacterial killing.
Peroxidation of lipids in the bacterial membranes or inside the cells, occurs as a chain reaction and results in the formation of harmful products, the most mutagenic and toxic being malondialdehyde (MDA) and 4-hydroxynonenal (4-HNE), respectively \cite{Ayala2014lipid}.
Attack on proteins is also harmful to the bacteria as proteins can be structurally altered, affecting their function.
For example, enzyme activity may be affected \cite{PhamHuy2008free}.
Finally, DNA attack by RONS can lead to harmful DNA mutations which can lead to cell death.
However, it is important to note that these effects have not been investigated in the context of plasma treatment specifically, more that these species are known to be able to cause these effects and it is possible that plasma-produced species are acting in the same way.

%Oxidative attack on lipids inside cells or as part of cell membranes results in a chain reaction and the formation of harmful products, the most mutagenic and toxic being malondialdehyde (MDA) and 4-hydroxynonenal (4-HNE), respectively \cite{Ayala2014lipid}.
%Lipid peroxidation occurs as a chain reaction, mediated by free radicals, with the hydroxyl radical ($\cdot$OH) being particularly effective.
%Firstly, the radical breaks an unsaturated carbon bond in the lipid, abstracts a hydrogen and forms a lipid radical, which then readily reacts with oxygen to form a lipid peroxy radical.
%The lipid peroxy radical can then attack another lipid and abstract a hydrogen, forming a new lipid radical and a lipid hydroperoxide. 
%Thus the process propagates until antioxidants such as vitamin E can neutralise the radical species \cite{Ayala2014lipid}.
%
%%Proteins can be changed structurally and enzymes break \cite{PhamHuy2008free}.
%
%As with lipid peroxidation, the $\cdot$OH radical is effective for damaging DNA. 
%There are many ways it can interact with the components of DNA, in particular through reactions with the purines and pyrimidines of DNA.
%Although the body has many DNA repair mechanisms in place, mutations as a result of the oxidative insult still occur.
%These are often result in mispairing of DNA bases and subsequent transversion mutations.
%For example, a common lesion formed in DNA is 8-OH-Gua (a guanine base that has been modified through attack by $\cdot$OH), which wrongly pairs with adenine and thus results in a G $\rightarrow$ T transversion mutation \cite{Dizdaroglu2012oxidatively}.
%%It readily reacts with nucleotide bases, which allows for frequent mutations 
%%DNA attack can cause mutations. \cite{PhamHuy2008free}.
%\todo{?DNA damage outcomes}


\subsubsection{UV}
UV emission is known to be harmful at certain wavelengths and intensities, and, as such, it has been used extensively for sterilisation purposes due to it's ability to interfere with DNA \cite{Laroussi2004evaluation}.
To investigate the effects of UV radiation emitted from plasmas, windows can be put between the plasma and the sample that allow specific wavelengths of UV radiation through, while blocking all other plasma components. 
Using these methods, there is evidence to suggest that the effects of UV radiation are negligible \cite{Laroussi2004evaluation, Dobrynin2009physical}.
%There is currently a lack of safety guidelines relating to the safe dose of UV that can be applied to skin.
%Whilst there are suggestions for undamaged skin, there is nothing for broken, damaged skin, therefore, keeping the UV emission as low as possible is probably advisable \cite{Isbary2013cold}.

\subsubsection{Charged Particles and Electric Fields}
During treatment, bacterial samples may or may not be exposed to charged particles and electric fields used to produce the plasma, depending on the configuration of the plasma device.
However, there is evidence to suggest that bacterial killing is more effective when they are in the treatment \cite{Fridman2007comparison}.
A proposed mechanism by which charged particles can contribute to bacterial killing came from Mendis \textit{et al} in 2000 \cite{Mendis2000a}.
It was suggested that accumulation of ions on the membranes of cells could cause an electrostatic force greater than the tensile strength of the membrane, causing rupture.
This was noted following the observations of Laroussi \textit{et al} \cite{Laroussi1999images} that showed morphological changes and lysis of the treated bacteria.
However, what is not clear is whether the effects seen by Laroussi \textit{et al} are due to ions, or due to the electric fields causing electroporation of the cells.
Electroporation occurs when the electric field across a cell membrane exceeds a certain threshold and results in pore formation in the membrane. If the applied voltage is too great, then this process can be irreversible and result in cell death through the inability to maintain a normal membrane potential using membrane ion transporters. 
This can lead to metabolic exhaustion of the cell, and death by necrosis \cite{Lee2006cell}.


%As charged particles are generally confined within the strong electric fields within the plasma, they don't tend to travel far outside the plasma bulk.
%This means that one method for delineating the roles of electric fields and particles from the other plasma components is by comparing direct (where the bacterial sample is in direct contact with the plasma) and indirect (where the bacterial sample is only exposed to the plasma effluent) plasma treatments.
%Investigations such as this have suggested that charged particles or electric fields, or a combination of both, are important for bacterial killing as direct treatment is much more efficient that indirect treatment \cite{Fridman2007comparison}.
%A proposed mechanism for this increased efficacy when charged particles are in contact with bacteria came from Mendis \textit{et al} in 2000 \cite{Mendis2000a} who suggested that accumulation of ions on the membranes of cells could cause an electrostatic force greater than the tensile strength of the membrane, causing rupture.
%This was noted following the observations of Laroussi \textit{et al} \cite{Laroussi1999images} that showed morphological changes and lysis of the treated bacteria.
%
%\todo{Electroporation occurs when the electric field across a cell membrane exceeds a certain threshold and results in pore formation in the membrane. If the applied voltage is too great, then this process can be irreversible and result in cell death through the inability to maintain a membrane potential using normal ion transporters in the membrane. Leads to metabolic exhaustion of the cell and death.
%Therefore, instead of the 'electrostatic disruption' theory, it may be that plasma is causing irreversible electroporation of the cells due to the electric fields rather than charged particles.}

\subsection{Host Cell Activation and Healing Promotion}
%It seems that the molecular interactions of plasmas are less well understood when considering the beneficial effects it has on host cells and tissues.
%However, by considering the normal, physiological effects of plasma components that are produced endogenously, and the observed outcomes from wound treatment with LTP, mechanisms of action can be proposed.
%
%It is generally accepted that low doses plasma treatment is good for stimulating effects - in the context of wound healing this would include things like increased proliferation and migration and DNA repair. 
%This contrasts with higher doses which generally induce damaging/lethal effects such as cell death and DNA damage \cite{Haertel2014nonthermal}. 

There is evidence to show that plasma treatment could accelerate wound healing, both from laboratory experiments, and clinical trials.
For example, following scratch tests, 3T3 keratinocytes treated with plasma recovered and re-grew to confluence more quickly than untreated controls \cite{Tipa2011plasma}.
This suggests that plasma treatment can increase keratinocyte proliferation, something which has also been shown by changes in gene expression in HaCaT keratinocytes following plasma treatment \cite{Barton2013nonthermal}.
Clinically, devices such as KinPen and MicroPlaSter have shown promise when used on human patients, aiding healing without any obvious side-effects \cite{Isbary2013cold, Isbary2012successful, Isbary2010a, Bekeschus2016the}.


%Plasma seems to be toxic to lymphocytes but not neutrophils or monocytes due to strong oxidation in the cell membrane and cytosol. This may be beneficial as excessive lymphocytes are present in pathological wounds and, therefore, by removing them, it may promote healing \cite{Bekeschus2016the}.
%\toref{Kramer2013suitability, Haertel2014, Bender2012, Brehmer2015, Joshi, Kong2009}

%Dealing with the groups of plasma components, there is proposed mechanisms of action for RONS and electric fields in wound healing.
%However, the actions of RONS and electric fields discussed below is not determined using plasma treatment, therefore, it can only be proposed that the same actions are carried out by plasma-produced species rather than endogenously produced species.


\subsubsection{RONS}
There is evidence suggesting that RONS are important at all stages of the normal, physiological wound healing process.
This begins with coagulation, where RONS are involved in the release of tissue factor which triggers the clotting cascade and in platelet recruitment and activation \cite{Soneja2005role}.
Plasma-produced RONS may also act similarly, shown by multiple studies demonstrating LTP can accelerate coagulation \cite{Fridman2006blood, Chen2009blood}.

%There is much evidence showing the importance of RONS in the normal, physiological process of wound healing.
%Throughout the different stages of healing, RONS have a role to play.
%Firstly, RONS are important for coagulation through involvement in the release of tissue factor (TF), which triggers the clotting cascade. RONS are also important for platelet recruitment and activation, subsequent further release of ROS and more TF production \cite{Soneja2005role}.
%It is possible that plasma derived RONS can also carry out this function, as shown by multiple studies of plasma on coagulation \cite{Fridman2006blood, Chen2009blood}.

RONS are also important for the inflammatory, re-epithelialisation and wound contraction stages of wound healing.
This is particularly due to their ability to act as signalling molecules and also their role in mediating the release of factors and receptor affinity.
These roles help control chemotaxis and adhesion of immune cells recruited to the wound site and help the activation and proliferation of keratinocytes to start re-epithelialisation of the wound \cite{Soneja2005role}.
RONS are also involved in collagen production and control of myofibroblast (the cells which contract to decrease the wound area) formation \cite{Soneja2005role}.


Nitric oxide (NO) is a very important species in wound healing and contributes to all the processes above \cite{Soneja2005role}.
The potential for using plasma-produced NO therapeutically for wound healing applications has been investigated.
For example Shekhter \textit{et al} \cite{Shekhter2005beneficial} used an air plasma device, named 'Plason', to produce an NO-containing gas flow which was then used to treat wounds in rats. 
The NO was produced by a high temperature electric arc discharge, and then the gas was allowed to cool before being used on the wounds. 
Wounds treated with 'plason' showed accelerated healing compared to the control group.
However, in this trial they attributed the positive effects purely to NO, without seemingly carrying out any characterisation of the plasma effluent they were using.
Being an air plasma, it is likely that there are other longer-lived species present in the effluent that could be contributing as well.

%Inflammation: Following immune cell recruitment to the wound site, they produce many factors important for potentiating the immune response. Species such as H$_2$O$_2$ are important secondary messenger molecules for these factors and are, therefore, important during an immune response. RONS in general are also important for affecting chemotaxis and adhesion of immune cells such as neutrophils and monocytes \cite{Soneja2005role}.
%
%Re-epithelialisation: ROS are important for promoting collagenase expression (for wound remodelling) and epidermal growth factor (EGF) expression, which promotes the proliferation of keratinocytes into the wound site \cite{Soneja2005role}.
%
%Angiogenesis: ROS also play a role in promoting angiogenesis, matrix deposition and wound contracture, by enhancing the release of pro-angiogenic factors and enhancing its receptor affinity, stimulating the release of collagen and controlling the transition of fibroblasts to myofibroblasts (the cells which contract to make the wound site smaller) \cite{Soneja2005role}.


\subsubsection{Electric Fields}

Normal cell membranes have a potential difference across them, due to the movement of ions across the membrane.
When epithelium is broken through wounding, the ion currents are disrupted and an electric field is established.
There is evidence to suggest that this electric field helps guide the migration of epithelial cells during healing \cite{Zhao2009electrical}.
Following this, there have been studies into the effects of using electric fields for the promotion of wound healing \cite{Thakral2013electrical, Messerli2011extracellular}, many of which showed promise.
However, the different studies used widely varying applied voltages/currents/waveforms and, therefore, comparison is difficult.

%However, as with plasma treatment, there is little understanding of the relation between 'dose' and outcomes etc, though looks promising.
%Also all the different trials use different regimes and therefore comparison is difficult.
%Lab trials show the importance of electrical currents/fields for wound healing, and there are clinical trials involving electric fields for wound healing that show promise \cite{Messerli2011extracellular}.
%\todo{Widely varying voltages and pulses etc so can't really compare.}
%\todo{E field strengths?? Applied voltages?? Comparison to plasmas}
%The threshold field strength noticeable by cells is 4500 times smaller than that required for electroporation of skeletal muscle \cite{Messerli2011extracellular}.

%\subsection{Plasma and the Immune Response}
%Plasma can stimulate macrophage migration \cite{Miller2014plasma}. 
%ns pulsed DBD plasmas used to treat macrophages in a scratch test. Macrophages grown to confluence then scratched with a pipette tip. Plasma treatment of macrophages increased the migration of the macrophages into the gap, but only at some frequencies. Lower frequencies of ns pulsing was better than higher frequencies which caused cell death.
%
%Cancer promotion of immunogenic cell death... However, wouldn't this be a bit concerning that normal host cells might also die like this and start autoimmune responses? For example, like the idea T1 diabetes starts with a virus?!

\section{Work to Date: Plasma Simulations}

For the last few months I have been continuing to work on plasma simulations, using GlobalKin.
As detailed in my previous TAP report, GlobalKin is a 0 dimensional global chemistry plasma model that calculates the time evolution of plasma species, given specific geometry and plasma parameters \cite{Stafford2004O2}.
To do this, it has an internal Boltzmann solver for calculating electron energy distribution functions over time which it then combines with a reaction chemistry module outlining all the species and reactions taking place in the plasma.
Finally, species densities are determined by an ordinary differential equation (ODE) solver to solve ODEs for species densities formed by the reaction chemistry module.
\textbf{Figure \ref{fig:GlobalKin}} is a schematic for how GlobalKin works.
For the chemistry module to work, an input file must be provided which contains all the plasma species and reactions along with a set of parameters.
For further description of the GlobalKin model, please refer to my 3 month TAP report.


\begin{figure}
\includegraphics[width=\textwidth]{Figures/GlobalKinDiagram}
\caption{Diagram outlining how GlobalKin works. Firstly, ODEs for species density and electron temperature are constructed. This takes direct GlobalKin inputs specified by the user, as well as electron reaction rate coefficients, diffusion coefficients and mobilities from the Boltzmann solver. An ODE solver then solves the equations and the new species densities are fed back into the ODEs and Boltzmann solver, for new densities to be calculated. At regular intervals, the Boltzmann solver also updates and new coefficients are fed back into the ODEs for the ODE solver to work with. This results in the time evolution of species densities and electron temperatures.}
\label{fig:GlobalKin}
\end{figure}



My project is concerned with the use of air plasmas, therefore, the aim is to develop an air reaction chemistry set for use in GlobalKin.
However, as air is highly complex, involving many species and reactions, I have started with developing a set for pure nitrogen plasmas, which currently includes 31 species and 207 reactions.


\subsection{Nitrogen Chemistry Set}
To begin the chemistry set, Vasco Guerra and Kinga Kutasi have kindly shared with us their air chemistry set used in \cite{Kutasi2016tuning}.
This gives a good starting point for the nitrogen reactions and their rates.

\subsubsection{Species}

The model so far consists of the 31 nitrogen species, in accordance with \cite{Kutasi2016tuning}.
This includes ground state N, N$_2$, N$^+$, N$_2^+$ and N$_4^+$; and various electronically excited states of N$_2$, N$_2^+$ and N.
The excited states are shown in table \ref{table:Species}.
Being a molecule, nitrogen can absorb energy into its bonds and is said to exist in multiple vibrational states.
For this chemistry set, the first 17 vibrational states of ground state nitrogen are included (N$_2$ (X, v = 0 - 17)).
The N$_2 (a/a'/w)$ states are also included, however, they are currently considered as a single state due to reasons discussed below.



\begin{table}
\caption{Excited States of species included in Nitrogen Chemistry Set}
\begin{center}
\begin{tabular}{| c | c |}
\hline
Species & States \\
\hline\hline \hline
N$_2$ & $X^1\Sigma_g^+  (v = 0 - 17),  A^3\Sigma_u^+, B^3\Pi_g, B'^3\Sigma_u^-, C^3\Pi_u, (a^1\Pi_g/a'^1\Sigma_u/w^1\Delta_u)$\footnotemark  \\
\hline
N$_2^+$ & $X^2\Sigma_g^+, B^2\Sigma_u^+ $ \\
\hline
N & $^4S, ^2D,  ^2P$ \\
\hline
%N$^+$ & - \\
%\hline
%N$_2^+$ & - \\
%\hline
%N$_4^+$ & - \\
%\hline
\end{tabular}
\end{center}
\label{table:Species}
\caption*{$^1$ These three states are considered as a single state by GlobalKin.}
\end{table}




\subsubsection{Reactions and Rates}

A list of all the possible reactions occurring in the plasma needs to be provided in an input file to GlobalKin.
Each reaction must be listed with it's rate or reaction cross section, which can be found in literature.

\subsubsection*{Electron Reactions}
The rates of electron impact reactions are highly dependent on electron energy, therefore, as electron energies in the plasma change, the rates change accordingly.
To account for this, electron impact cross sections are specified instead of a rate.
The reaction cross section is the probability of a reaction occurring given an electron is at a particular energy.
GlobalKin can then use the cross sections alongside the Boltzmann solver (which calculates the electron energy distribution function at regular intervals) to calculate the reaction rate at regular time points.

GlobalKin contains the electron impact cross sections for most reactions internally and for the time being, only electron impact reactions with cross sections listed in GlobalKin have been included in this chemistry set.
For this reason, the three N$_2$ states of N$_2$(a), N$_2$(a') and N$_2$(w), are all considered as a single state as there is only one cross section listed for all three of these species in GlobalKin.

\subsubsection*{Heavy Particle Reactions}
Unlike electrons, heavy particles do not gain significant energy from the electric field in the plasma, due to their higher mass.
For this reason, reaction rates can be stated as constants, in the Arrhenius equation format.
These rates are found in the literature.


\subsubsection*{Vibrational Kinetics of Nitrogen}

As shown above, nitrogen has many vibrational states, the first 17 of which are included in this chemistry set.
Vibrational states are important to include in nitrogen models due to the fact that these states have an impact on the electron energy distribution function (EEDF) \cite{Guerra2004kinetic}.
This is due to there being a high probability of electrons colliding with N$_2$ molecules and exciting them to different vibrational states (equation \ref{eqn:ElectronVibration}), causing electron energy losses.
\begin{equation}
e^- + N_2(X,v=0) \rightarrow N_2(X,v=1) + e^-
\label{eqn:ElectronVibration}
\end{equation}

Also, superelastic collisions of electrons with vibrational states (equation \ref{eqn:SuperelasticVibrational}) cause electrons to gain energy.

\begin{equation}
e^- + N_2(X,v=1) \rightarrow N_2(X,v=0) + e^-
\label{eqn:SuperelasticVibrational}
\end{equation}

As well as for their impact on the EEDF, vibrational states can also have an impact on the dissociation of N$_2$, and on gas heating in the plasma \cite{Guerra2004kinetic}.
For this reason it is important to include the production and destruction mechanisms of vibrational states.
In particular, there are two types of reactions involving vibrationally excited N$_2$ that are included in the chemistry set.
The first is vibration-translation (V-T) reactions. In these reactions energy from vibrational excitation is transferred into other forms of energy, such as heat, and the reaction can, therefore, contribute to gas heating.
V-T reactions can occur between vibrationally excited N$_2$ and either N$_2$ molecules or N atoms, as follows:

\begin{equation}
N_2(X,v=n) + N_2 (X,v=0) \rightarrow N_2(X,v=n-1) + N_2(X,v=0)
\label{eqn:V-TN2}
\end{equation}
\begin{equation}
N_2(X,v=n) + N \rightarrow N_2(X,v=n-1) + N
\label{eqn:V-TN}
\end{equation}

The second type of reaction is vibration-vibration (V-V) reactions, where vibrational energy is transferred between N$_2$ molecules as follows:

\begin{equation}
N_2(X,v=w) + N_2 (X,v=n) \rightarrow N_2(X,v=w-1) + N_2(X,v=n+1)
\label{eqn:V-V}
\end{equation}

Dissociation of N$_2$ is also influenced by the presence of vibrational states of N$_2$, however, this tends to be at the higher vibrational states (e.g. n = 45) \cite{Guerra2004kinetic}, which are not included in this model at the moment. 
The importance of these higher vibrational states will need to be investigated and could be added in at a later date.


\subsubsection{Rates of Reactions Involving Vibrational States}

As with other heavy particle reactions, the rates for these reactions are found in the literature.


The rates for N$_2$-N$_2$ V-T reactions (\textbf{equation \ref{eqn:V-TN2}}) can be found in  \cite{Billing1979vv} for n = 1 - 9 \& 20.
Therefore, it was necessary to fit the data to determine rate coefficients for n = 10 - 17 required for the model.
Exponential, 3rd degree polynomial and linear fits were tested.
These are shown in \textbf{figures \ref{fig:fits}}.
As shown in figures \ref{subfig:Exp} and \ref{subfig:ExpAll}, the exponential fit only worked when n=20 was excluded.

To then check how much difference the fits made to the rates and, therefore, the densities of the vibrational states of N$_2$, the model was run twice, once with rates from the linear fit, and once with rates from the polynomial fit.
From this, vibrational distribution functions (VDFs) were then plotted to see how different they were.
To plot the VDFs, the vibrational state densities are first normalised to the ground state density and then the VDF shows the populations of vibrationally excited N$_2$ compared to the ground state.
Alongside this, the density of atomic nitrogen was also determined to see how this was affected.
The results are shown in \textbf{figure \ref{fig:VDFandN}}.
As shown in the figures, which fit is used makes no difference to either the VDF or the atomic nitrogen density.
In \textbf{figure \ref{subfig:VDF}}, the VDFs for both fits are shown, however, only one line is visible as it is directly on top of the other.
It was, therefore, decided that the linear fit data would be used for the model.



%\begin{table}
%\begin{center}
%\caption{Rates for V-T reactions from \cite{Billing1979vv}}
%\begin{tabular}{| c | c | c | c | c | c | c | c | c | c | c |}
%\hline
%n & 1 & 2 & 3 & 4 & 5 & 6 & 7 & 8 & 9 & 20 \\
%\hline
%Rate ($ \times 10^{-21}$) &0.8 & 1.8 & 3.1 & 5.0 & 7.4 & 11 & 16 & 26 & 38 & 1600 \\
%\hline
%\end{tabular}
%\label{table:BillingRates}
%\end{center}
%\end{table}




\begin{figure}
\begin{subfigure}{0.5\textwidth}
\begin{center}
\includegraphics[width=\textwidth]{Figures/ExpFit}
\caption{Exponential Fit for n=1-9}
\label{subfig:Exp}
\end{center}
\end{subfigure}
\begin{subfigure}{0.5\textwidth}
\includegraphics[width=\textwidth]{Figures/ExpFitAll}
\caption{Exponential Fit for n=1-20}
\label{subfig:ExpAll}
\end{subfigure}
\begin{subfigure}{0.5\textwidth}
\includegraphics[width=\textwidth]{Figures/Linear}
\caption{Linear fit between n=9 and n=20}
\end{subfigure}
\begin{subfigure}{0.5\textwidth}
\includegraphics[width=\textwidth]{Figures/Polynomial}
\caption{Polynomial Fit for n=1-20}
\end{subfigure}
\caption{Figure Showing Fitting to V-T reaction rates from Billing and Fisher, 1979 \cite{Billing1979vv}}
\label{fig:fits}
\end{figure}

\begin{figure}
\begin{subfigure}{0.5\textwidth}
\includegraphics[width=\textwidth]{Figures/Fit_VDF_Comparison}
\caption{Vibrational Distribution Function}
\label{subfig:VDF}
\end{subfigure}
\begin{subfigure}{0.5\textwidth}
\begin{center}
\includegraphics[width=0.7\textwidth]{Figures/Ndensity}
\caption{Atomic Nitrogen Density}
\end{center}
\end{subfigure}
\caption{Graphs showing the vibrational distribution functions (a) and atomic nitrogen densities (b) when different fits are applied to V-T reaction rate data in \cite{Billing1979vv}}
\label{fig:VDFandN}
\end{figure}

Rates for N$_2$-N V-T reactions \textbf{equation \ref{eqn:V-TN}} were taken from \cite{Guerra1995non}.
These rates are being used for the moment, however, Guerra \textit{et al} \cite{Guerra2004kinetic} suggest that they may be too high.
Therefore, it is something to consider changing in future iterations of the model.

%
%Rates for V-V reactions where n = 1 - 9 and n = 20 (\textbf{equation \ref{eqn:V-V}}) were also taken from \cite{Billing1979vv}, with subsequent data fitting.
%For these rates, only a linear fit between n = 9 and n = 20 worked, therefore, this is how rates were determined for n = 10 - 17.
%Fitting is shown in \textbf{\ref{fig:VVFit}}.

%\begin{figure}
%\begin{center}
%\includegraphics[width=0.7\textwidth]{Figures/VVFitting}
%\caption{V-V Reaction Rate Fitting using data from \cite{Billing1979vv}}
%\label{fig:VVFit}
%\end{center}
%\end{figure}


%---------------
%This is the bit about V-T reactions. Maybe change if redo simulations....
%---------------
%To investigate the importance of inclusion of the V-T reactions discussed above, the VDF and N densities were investigated when the chemistry included V-T reactions, and when they were removed.
%The results of this are shown in figure \ref{fig:VT_vs_no_VT}.
%The figure shows that the addition of V-T reactions has little/no effect on the VDF or N density.
%However, there is slight divergence between the two at the highest vibrational states.
%This not surprising as it is thought that V-T reactions are thought to be most important at higher vibrational states \cite{Guerra2004kinetic}.
%
%\begin{figure}
%\begin{subfigure}{0.5\textwidth}
%\begin{center}
%\includegraphics[width=\textwidth]{Figures/VDF_w_wo_VT}
%\caption{Vibrational Distribution Function}
%\label{fig:VDFComp}
%\end{center}
%\end{subfigure}
%\begin{subfigure}{0.5\textwidth}
%\begin{center}
%\includegraphics[width=0.84\textwidth]{Figures/N_density_w_wo_VT}
%\caption{Atomic Nitrogen Density}
%\label{fig:NComp}
%\end{center}
%\end{subfigure}
%\caption{Comparison of VDF (a) and N density (b) when V-T reactions are included or excluded from the reaction chemistry set.}
%\label{fig:VT_vs_no_VT}
%\end{figure}

As discussed earlier, V-V reactions may also be important to include in the chemistry set.
I am currently in the process of finding rates for these reactions and adding them into the chemistry set.

%The impact of including these V-V and V-T reactions was then tested by comparing the VDF when both reactions were included compared to only one or the other. 
%The rest of the reaction set remained unchanged.
%The VDFs are shown in figure \ref{fig:VDFComp}.
%From this, it can be seen that the whether or not V-T reactions are included have very little impact on the VDF (red and green lines).
%However, there is slight divergence between the two at the highest vibrational states.
%This not surprising as it is thought that V-T reactions are thought to be most important at higher vibrational states \cite{Guerra2004kinetic} \todo{check the paper for this!}
%\future{Check to look at influence on gas temperature}
%
%V-V reactions, however, change the shape of the VDF more obviously.
%The shape seems to change quite significantly, particularly at the lowest vibrational states.
%This is interesting as it suggests it may be important to add in more of these reactions.
%Currently only V-V reactions where $w=1$ are included, though this result suggests it may be important to add more V-V reactions where $w > 1$.
%
%The impact of having both V-V and V-T reactions included in the model is shown by the comparison of the VDF both with and without V-V reactions.
%This is shown in \textbf{\ref{fig:VDFComp}}.
%At the moment, in the V-V reactions included in the chemistry set, w=1.
%\todo{I think V-V reactions can go from w = other numbers...!}
%
%\begin{figure}
%\begin{subfigure}{0.5\textwidth}
%\begin{center}
%\includegraphics[width=\textwidth]{Figures/VDF_w_wo_VT}
%\caption{Comparing the VDF with and without V-V reactions included in the reaction set}
%\label{fig:VDFComp}
%\end{center}
%\end{subfigure}
%\begin{subfigure}{0.5\textwidth}
%\begin{center}
%\includegraphics[width=\textwidth]{Figures/N_density_w_wo_VT}
%\caption{Comparing the VDF with and without V-V reactions included in the reaction set}
%\label{fig:NComp}
%\end{center}
%\end{subfigure}
%\end{figure}


\subsection{Model Outcomes and Reaction Pathways Analysis}

As an example of how the model will be useful alongside experiments in the future, figure \ref{fig:ExamplePowerVar} shows how species densities can be predicted by the model, when plasma parameters are altered.
Specifically, using the chemistry set introduced above, densities of atomic nitrogen and the excited state N$_2$(A) are shown for different plasma powers.
N$_2$(A) is a long-lived (metastable) excited state of N$_2$ with high energy ($\sim$ 6 eV), meaning it is reactive and could therefore have an effect on biological samples.
Interestingly, the model shows that as plasma power is increased, the density of atomic nitrogen increases, whereas the N$_2$(A) density doesn't change.
This gives an indication that the two species could be controlled independently and, therefore, could be useful to control biological effects.

\begin{figure}
\includegraphics[width=\textwidth]{Figures/ExampleDensities}
\caption{Power Variation}
\label{fig:ExamplePowerVar}
\end{figure}


As an extension of GlobalKin, there is also a reaction pathways analysis tool, PumpKin \cite{Markosyan2014pumpkin}, which is capable of indicating the dominant reaction pathways causing the production and destruction of species.
Briefly, it works by taking a specified short timescale and identifying fast-lived species that are recycled on a timescale shorter that that.
Reactions containing these species are then combined so that there is no net production or destruction, and by doing so, reaction pathways are formed which can help the investigation of slower chemical dynamics, involving species of interest.
As an example, PumpKin analysis was performed on the production of N in the power variation shown in figure \ref{fig:ExamplePowerVar}, to look at the main N production pathways and see if this changes with power variation.
The results of this analysis are shown in figure \ref{fig:pumpkin}.
The results show that there are 4 reaction pathways responsible for N production:
\begin{equation}
N_2(X, v=10) + N_2(X, v=10) \rightarrow N_2 + N + N
\label{eqn1}
\end{equation}
\begin{equation}
N(^2P) + N_2 \rightarrow N + N_2
\label{eqn2}
\end{equation}
\begin{equation}
N_2(X, v=11) + N_2(X, v=11) \rightarrow N_2 + N + N
\label{eqn3}
\end{equation}
\begin{equation}
e^- + N_2 \rightarrow  N + N + e^-
\label{eqn4}
\end{equation}

Interestingly, as the plasma power is increased, the contribution to N production by equation \ref{eqn1} and, to a lesser extent equation \ref{eqn3} increases, whereas the contribution by equations \ref{eqn2} and \ref{eqn4} decreases.
This gives information about the reaction dynamics that is not available experimentally.
This result is in agreement with that shown by Guerra \textit{et al} \cite{Guerra2004kinetic}, who states that $N_2(X, 10<v<25) + N_2(X, 10<v<25) \rightarrow N_2 + N + N $ is an important dissociation pathway, particularly as pressure increases.

\begin{figure}
\includegraphics[width=\textwidth]{Figures/N_Production_PumpKin}
\caption{PumpKin example}
\label{fig:pumpkin}
\end{figure}

It is important to note that these simulations are currently not benchmarked, therefore, the model is not predictive as yet.
However, the main step in the coming months is to begin the benchmarking process in order to check both that the chemistry is working as it should and that the results of the model are starting to agree with experimental results.

\section{Future Directions}
\subsection{Simulations}
The future directions for the simulations are mainly relating to model refinement.
For example, V-V reactions are currently not included, but are probably required to more accurately model nitrogen plasmas.
%For example, it is necessary to investigate whether there are reactions missing from the reaction set that need to be added.
%For instance, there may be further reactions involving vibrational states that are currently not included (such as V-V reactions where $w > 1$), but are required to make the reaction chemistry set realistic.
%To achieve this, further literature search is required to understand the importance of other reactions involving vibrational states of N$_2$.

Alongside this, experimental benchmarking of the model is required to see how well simulated species densities match experimental densities.
Therefore, the type of plasma source to be used needs to be decided on.

\subsection{Plasma Source}
When considering the geometry of the plasma source to be used, it is important to look at the requirements for the source.
The main criteria are:
\begin{itemize}
\item The ability to carry out diagnostics on the plasma source in order to be able to characterise it properly, so we can understand the plasma being produced. 
\item As part of the characterisation, being able to model the plasma would be useful. For using GlobalKin, therefore, it would be beneficial to have a homogeneous plasma, with a well defined plasma volume
\item Be able to treat different biological samples, such as cells or tissues
\end{itemize}

There are multiple different geometries that have been used for air plasmas, such as plasma jets \cite{Pei2012inactivation, Chen2009blood, Walsh2011portable}, corona discharges \cite{Dobrynin2011inactivation} and dielectric barrier discharges (DBDs).
Whilst all these different geometries have their advantages and disadvantages, one major disadvantage of the likes of jets and corona discharges is that it is difficult to model them quantitatively in GlobalKin.
This is due to two reasons.
Firstly, the plasma volume is less well defined and secondly, the plasmas are often not very uniform and often filamentary.
Therefore, a DBD configuration might be the most appropriate for the initial experimental stages of the project, even if the design requires changing in the future so that it is better for application to actual wounds.


\subsubsection{Dielectric Barrier Discharge}
Dielectric barrier discharges are so called due to having a dielectric barrier on one of the electrodes to limit the current flow and prevent the glow-to-arc transition. 
They can operate at atmospheric pressure using air and don't necessarily require a gas flow \cite{Fridman2013plasmamedicine}
They have been used extensively for applications such as ozone production and polymer treatment and more recently, for biomedical applications \cite{Fridman2013plasmamedicine, Brehmer2015alleviation}.
A typical DBD setup used to treat a biological sample is shown in figure \ref{fig:DBD}.
DBDs have shown to be effective for bacterial killing and other biomedical applications such as blood clotting, and there have been clinical trials into their effectiveness for wound healing \cite{Daeschlein2012in, Fridman2006blood, Brehmer2015alleviation}. 

In terms of operation, there was a study investigating pulsing of DBD, using both micro- and nanosecond pulses, in particular, investigating the homogeneity of the plasma produced. 
It was found that nanosecond pulsing gave a more homogenous plasma, that would ignite more uniformly over a non-uniform surface, and was more effective for bacterial killing than microsecond pulsing \cite{Ayan2009application, Ayan2008nanosecond}.



%DBDs have to be operated using alternating current.


%\todo{Speak with Jerome about air plasma designs. Something Sandra built to try with his nanosecond supply??}


\begin{figure}
\centering
\includegraphics[width=0.5\textwidth]{Figures/DBD}
\caption{DBD}
\label{fig:DBD}
\end{figure}

%DBDs have been shown in laboratory conditions to be effective for bacterial killing \cite{Daeschlein2012in, Fridman2006blood}.
%An air DBD, PlasmaDerm has been tested on chronic ulcers in a clinical trial, though the results are not overly promising \cite{Brehmer2015alleviation}, as plasma induced killing of wound colonising bacteria was not long lasting and there did not appear to be any acceleration in wound healing in the plasma treated group compared to the controls.
%The device operates at AC voltage pulses with amplitudes $>$10 kV and a power density of 120 mW/cm\textsuperscript{2} \cite{Brehmer2015alleviation}.



%\cite{Fridman2007comparison} looks at direct and indirect plasma treatment. Using DBD with and without mesh/airflow to try and separate out the roles of ions and reactive species.
%Both use atmospheric air as the gas.
%Both operate at 35 kV, 12 kHz frequency and power density of 0.8 W/cm\textsuperscript{3}.


%Specifically, patterned (i.e. not flat) agar was used as the target and it was seen that with microsecond pulses, the plasma was more filamentary and some areas didn't ignite at all \cite{Ayan2009application}. 
%However, using the nanosecond pulses, the plasma was more uniform over the entire surface.
%This is preferable when considering wound treatments, as the wound bed is not smooth.
%Further to this, nanosecond pulsed DBD was found to be more effective for bacterial killing \cite{Ayan2008nanosecond}.

%
%\subsubsection{Floating-Electrode Dielectric Barrier Discharge}
%Similar to DBD, the FE-DBD can be used whereby one of the electrodes can be replaced with anything with a sufficiently high charge storage capacity, a so-called, floating electrode (FE). 
%Biological samples and skin are able to do this due to their high water content {\cite{Fridman2006blood}}.
%Good diagrams of the electrodes are shown in {\cite{Fridman2006blood}}.
%Has been shown that FE-DBD can accelerate the coagulation process in the blood. It seems to catalyse the physiological process of clotting rather than exerting a physical influence.

\section{Conclusions}

The work to date leads to the next immediate goals as follows:
\begin{itemize}
\item Continue working on the nitrogen chemistry set, making sure required reactions are included. Subsequently, begin validation of the chemistry by comparing simulated results with experimental measurements
\item Decide on the type of plasma source, bearing in mind the suitability for application, characterisation and plasma modelling
\item Begin thinking about the biological questions to be answered and the types of biological assays to be used
\end{itemize}

\scriptsize
\bibliographystyle{ieeetr}
\bibliography{/Users/hld523/Bibliography/MyPapers}
\end{document}  