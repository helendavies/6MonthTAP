\documentclass[11pt, oneside]{article}   	% use "amsart" instead of "article" for AMSLaTeX format
\usepackage{geometry}                		% See geometry.pdf to learn the layout options. There are lots.
\geometry{letterpaper}                   		% ... or a4paper or a5paper or ... 
%\geometry{landscape}                		% Activate for rotated page geometry
%\usepackage[parfill]{parskip}    		% Activate to begin paragraphs with an empty line rather than an indent
\usepackage{graphicx}				% Use pdf, png, jpg, or eps§ with pdflatex; use eps in DVI mode
								% TeX will automatically convert eps --> pdf in pdflatex		
\usepackage{amssymb}

%SetFonts

%SetFonts


\title{6 Month TAP Report}
\author{Me}
%\date{}							% Activate to display a given date or no date

\begin{document}
\maketitle
\section{Introduction}
\subsection{Overview of previous TAP}
Here are the aims that I set in last TAP meeting:
\begin{itemize}
\item Decide on air chemistry set
\item Decide on air plasma geometry
\end{itemize}

Not part of collaboration any more.

\subsection{Motivation}
Here I will talk about air plasmas and lots of biology. 
Lots of examples of what plasmas, in particular air plasmas, can do. 
That'll keep the biologists happy.


\section{Air Plasma Source Design}
\subsection{Confiuration}
Testing citation: \cite{Kolb2008cold}.

Overview of what others have done. Can use some pictures from the 1.2.17 lab meeting presentation I did.
Pictures in the emails from Deborah.

\subsection{Power Supply}
Nanosecond pulsing.
Why? What is the point? What does it mean? Why is it necessary for air and less so for noble gas plasmas?

\section{Air plasma model}
Whatever needs to go in here.... Vasco Guerra's air chem set....

\bibliographystyle{ieeetr}
\bibliography{/Users/hld523/Bibliography/MyPapers}
\end{document}  