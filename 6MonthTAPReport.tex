\documentclass[11pt, oneside]{article}   	% use "amsart" instead of "article" for AMSLaTeX format
\usepackage{geometry}                		% See geometry.pdf to learn the layout options. There are lots.
\geometry{letterpaper}                   		% ... or a4paper or a5paper or ... 
%\geometry{landscape}                		% Activate for rotated page geometry
%\usepackage[parfill]{parskip}    		% Activate to begin paragraphs with an empty line rather than an indent
\usepackage{graphicx}				% Use pdf, png, jpg, or eps§ with pdflatex; use eps in DVI mode
								% TeX will automatically convert eps --> pdf in pdflatex		
\usepackage{amssymb}
\usepackage{color}
\newcommand{\todo}[1]{ \textcolor{red}{\bf{To Do:} #1}}
\newcommand{\toref}[1]{ \textcolor{blue}{\bf{REFERENCE #1}}}
%SetFonts

%SetFonts


\title{6 Month TAP Report}
\author{By Me :)}
\date{}							% Activate to display a given date or no date

\begin{document}
\maketitle

\section{Wounds and Plasma Treatment}
Plasmas are rapidly extending into the field of medicine, with potential applications for wound healing \cite{Haertel2014nonthermal, Isbary2013nonthermal}, cancer treatment \cite{Hirst2016low, Fridman2007floating}, coagulation \cite{Fridman2006blood, Chen2009blood} and sterilisation \cite{Fridman2006blood, Laroussi2002nonthermal}.
Applications such as there pertain to plasmas ability to interact with living biological substrates such as bacteria, both in planktonic and biofilm forms \cite{Joshi2010control, Pei2012inactivation, Ziuzina2015cold} and eukaryotic, mammalian cells \cite{Haertel2014nonthermal}.
For sterilisation applications, the properties of low temperature plasma, in particular the low temperature and ability to operate at atmospheric pressure, allow their use both on equipment surfaces \cite{Laroussi2002nonthermal} and also directly on biological tissue, for example chronically infected wounds \cite{Haertel2014nonthermal, Isbary2013nonthermal, Fridman2006blood}.

This project is concerned with wound healing and how low temperature air plasma can be developed to allow treatment of wounds.
Plasmas can be used on wounds with two aims:
\begin{enumerate}
\item Bacterial killing
\item Host cell activation and healing promotion
\end{enumerate}

While exact mechanisms of action are not fully understood, there have been investigations into the roles of the different plasma components.
Briefly, low temperature plasmas are very weakly ionised gases (degree of ionisation $<$ 1\%), with a global temperature of of approximately room temperature ($< 40^{\circ} C$), and dry reactive environment, allowing direct contact with biological substrates.
Electrons liberated in the plasma mediate the different plasma components, for example, dissociation of molecules, yielding reactive radical species, emission as a result of electron impact electronic excitation and ionisation which yields ions and allows plasma sustainment.
It is, therefore, likely that the efficacy seen so far using plasmas is due to this complex plasma composition, and also what suggests that bacteria are going to be less likely to build up resistance, due to the multimodal nature of the plasma treatment.

The main three components that will be discussed in the next sections are charged particles and associated electric fields, emission of, for example UV photons, and reactive neutral species.


\subsection{Bacterial Killing}
\subsubsection{RONS}
Whilst reactive species were thought to be purely damaging molecules produced in the body which only serve to promote disease and ageing \cite{Harman1955aging}, it is now known that their role is far more complex than this, and that reactive species have crucial roles to play in the immune system and in normal physiological signalling \cite{Thannickal2000reactive}.
RONS are produced by phagocytes during an immune response in order to kill off invading bacteria, and severe diseases such as chronic granulomatous disease (CGD) can develop if part of the system generating these RONS is deficient \cite{Fang2004antimicrobial}
In order to cause bacterial killing, RONS can interact with multiple bacterial targets such as thiols, metal centres, DNA and lipids \cite{Fang2004antimicrobial}.
ROS in particular are attributed to DNA damage, whereas RNS inhibit respiration and interfere with DNA replication \cite{Fang2004antimicrobial}.

LTP is a highly efficient source of reactive species, produced by mechanisms such as electron impact excitation and dissociation.
The gas present in the plasma determines the types of species that will be produced. 
In the case of air plasma, reactive oxygen and nitrogen species (RONS) are readily produced and these have multiple possible effects on biological substrates.
Important species produced include atomic oxygen (O), ozone (O$_3$), singlet delta oxygen (O$_2$($^1\Delta$)), superoxide (O$_2$-), atomic nitrogen (N), hydroxyl radicals (OH), hydrogen peroxide (H$_2$O$_2$) and various nitrogen oxides (NO, NO$_2$ and N$_2$O) \cite{Graves2014low}.


\subsubsection{UV}
UV emission is known to be harmful at certain wavelengths and powers, and, as such, it has been used extensively for sterilisation purposes due to it's ability to interfere with DNA \cite{Laroussi2004evaluation}.
Guidelines state that \todo{find paper with safety guidelines in}.
To investigate the effects of UV radiation emitted from plasmas, windows can be put between the plasma and the sample that allow specific wavelengths of UV radiation through, while blocking all other plasma components. 
Using these methods, there is evidence to suggest that the effects of UV radiation are negligable \cite{Laroussi2004evaluation, Dobrynin2009physical}.

\subsubsection{Charged Particles and Electric Fields}
As charged particles are generally confined within the strong electric fields within the plasma, they don't tend to travel far outside the plasma bulk.
This means that one method for delineating the roles of electric fields and particles from the other plasma components is by comparing direct (where the biological substrate is in direct contact with the plasma) and indirect (where the biological substrate is only exposed to the plasma effluent) plasma treatments.
Investigations such as this have suggested that charged particles or electric fields, or a combination of both, are important for bacterial killing as direct treatment is much more efficient that indirect treatment \cite{Fridman2007comparison}.
A proposed mechanism for this increased efficacy when charged particles are in contact with the biological sample came from Mendis $et al$ in 2000 \cite{Mendis2000a} who suggested that accumulation of ions on the membranes of cells could cause an electrostatic force greater than the tensile strength of the membrane causing rupture.
This was noted following the observations of Laroussi \textit{et al} \cite{Laroussi1999images} that showed morphological changes and lysis of the treated bacteria.
\todo{Look into electroporation mechanism. Is this just electroporation by plasmas???}

\subsection{Host Cell Activation and Healing Promotion}
\todo{?Include anything on safety}

Plasma treatment increased speed of confluence after scratch test \cite{Tipa2011plasma}.
Plasma treatment of keratinocytes increases their proliferation and the production of growth factors \cite{Bekeschus2016the} (which sites other papers).
\cite{Barton2013nonthermal} looks at the effect on gene expression of HaCaT cells following plasma treatment in relation to genes important in wounding/healing.
Plasma seems to be toxic to lymphocytes but not neutrophils or monocytes due to strong oxidation in the cell membrane and cytosol. This may be beneficial as excessive lymphocytes are present in pathological wounds and, therefore, by removing them, it may promote healing \cite{Bekeschus2016the}.
\toref{Kramer2013suitability, Haertel2014, Bender2012, Brehmer2015, Joshi, Kong2009}


Activation of host cells..... 

\subsubsection{RONS}
\subsubsection{UV}
\subsubsection{Charged Particles and Electric Fields}



\section{Plasma Sources}
Here I talk about geometries that I could use, specifically review air plasmas.
\subsection{Geometry}
\subsubsection{Voltages, Waveforms and Power etc}

Nanosecond pulsing.
Why? What is the point? What does it mean? Why is it necessary for air and less so for noble gas plasmas?


\section{Work to Date: Plasma Modelling}
Use of GlobalKin, as presented in the previous TAP, this is a 0-Dimensional, global plasma chemistry model which calculates time evolution of plasma species, given specific geometry and plasma parameters.
For the last TAP I was just using Sandra's chemistry set to learn how to use the model and try to simulate some of the experiments I carried out during my rotation project.
However, since then I have been working towards building my own air chemistry set.
This is not a trivial task and therefore, I am starting with building a set for nitrogen.

\subsection{Nitrogen Chemistry Set}
Ta to Vasco Guerra for set from \cite{Kutasi2016tuning}.
\subsubsection{Species}
Included species included in Vasco: N, N$_2$, the first 17 vibrational states of ground state N$_2$ (N$_2$ (X, v = 0 - 17)), excited states of N$_2$ (N$_2$ (A), N$_2$ (B), N$_2$ (B'), N$_2$ (C), and a species N$_2$ (AP1S) which is the combination of N$_2$ (a), N$_2$ (a') and N$_2$ (w) (GlobalKin only contains cross sections for this collective species, rather than each of the three individual ones, therefore, for the time being, they have to be treated as one), excited N atoms (N ($^2$P) and N($^2$D) and ions N$_2$+, N$_4$+ and N$_2$ (B)+.
\subsubsection{Electron Impact Reactions}
In-built cross sections in GlobalKin
\subsubsection{VT reactions}
These are important reactions as they are important destruction mechanisms for vibrationally excited nitrogen molecules

\section{Future Work}
\begin{itemize}
\item Add to chemistry set by adding in O$_2$. For now, I am only going to consider dry air, therefore only N$_2$:O$_2$ gas mixture.
\item EXPERIMENTS!!!!
\end{itemize}




\bibliographystyle{ieeetr}
\bibliography{/Users/hld523/Bibliography/MyPapers}
\end{document}  